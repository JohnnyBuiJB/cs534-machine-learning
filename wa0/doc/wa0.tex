\documentclass[12pt,article]{article}
\usepackage{fullpage}
\usepackage[top=2cm, bottom=4.5cm, left=2cm, right=2cm]{geometry}
\usepackage{amsmath,amsthm,amsfonts,amssymb,amscd}
\usepackage{lastpage}
\usepackage{enumerate}
\usepackage{fancyhdr}
\usepackage{mathrsfs}
\usepackage{xcolor}
\usepackage{graphicx}
\usepackage{listings}
\usepackage{hyperref}
\usepackage{mdframed}
\usepackage{changepage}   % for the adjustwidth environment
\usepackage{forest} 
\usepackage{tikz}   % For graph

\usepackage{float}  % To inforce inserting images at the right place


\newcommand{\Tau}{\mathrm{T}}


% For matrix
\def\horzbar{\text{magic}}

\hypersetup{%
  colorlinks=true,
  linkcolor=blue,
  linkbordercolor={0 0 1}
}

\setlength{\parindent}{0.0in}
\setlength{\parskip}{0.05in}

\newcommand\projnumber{0}
\newcommand\course{CS534}
\newcommand\OSUID{934370552}
\newcommand\Email{buivy@oregonstate.edu}
\newcommand\Name{Vy Bui}
\newcommand\tab[1][1cm]{\hspace*{#1}}

\pagestyle{fancyplain}
\headheight 35pt
\lhead{Homework Assignment \projnumber}
\rhead{Sept. 29, 2022}
\lfoot{}
\cfoot{}
\rfoot{\small\thepage}
\headsep 1.5em

\newenvironment{problem}[2][Problem]
    { \begin{mdframed}[backgroundcolor=gray!20] \textbf{#1 #2} \\}
    {  \end{mdframed}}
   

% Make Rightarrow with superscript
\makeatletter
\newcommand{\xRightarrow}[2][]{\ext@arrow 0359\Rightarrowfill@{#1}{#2}}
\makeatother

\begin{document}

\begin{titlepage}
    \begin{center}
        \vspace*{4cm}

        \textbf{\Large CS534 - Machine Learning}

        \vspace{0.5cm}
 
        \textbf{\Large Homework Assignment \projnumber{}}
 
        \vspace{1cm}

        Author: Vy Bui

        OSUID: 934370552 \\ 
        
        Email: buivy@oregonstate.edu

        \vfill
             
        \vspace{0.8cm}
      
             
        The School of Electrical Engineering and Computer Science\\
        Oregon State University\\
             
    \end{center}
\end{titlepage}

%==============================================================================%
\newpage
\section*{Linear Algebra}
\textbf{(1a)} The product is not defined because the neighboring dimensions do 
not match.

\textbf{(1b)} 
\begin{align*}
    \left[\begin{array}{ccc}
        1 & 2 & 3 \\
        4 & 5 & 6 \\
        7 & 8 & 9 \\
    \end{array}\right]
    \begin{bmatrix}
        1 & 1 & 0 \\
        0 & 1 & 1 \\
        1 & 0 & 1 \\
    \end{bmatrix}
    {} &=
    \begin{bmatrix}
        1\times1 + 2\times0 + 3\times1 & 1\times1 + 2\times1 + 3\times0 & 1\times0 + 2\times1 + 3\times1 \\
        4\times1 + 5\times0 6\times1 & 4\times1 + 5\times1 + 6\times0 & 4\times0 + 5\times1 + 6\times1 \\
        7\times1 + 8\times0 + 9\times1 & 7\times1 + 8\times1 + 9\times0 & 7\times0 + 8\times1 + 9\times1 \\
    \end{bmatrix}
    \\ {} &=
    \begin{bmatrix}
        4 & 3 & 5 \\
        10 & 9 & 11 \\
        16 & 15 & 17 \\
    \end{bmatrix}
\end{align*}

\textbf{(1c)}
\begin{align*}
    \left[\begin{array}{cccc}
        1 & 2 & 1 & 2 \\
        4 & 1 & -1 & -4 \\
    \end{array}\right]
    \begin{bmatrix}
        0 & 3 \\
        1 & -1 \\
        2 & 1 \\
        5 & 2 \\
    \end{bmatrix}
    {} &=
    \begin{bmatrix}
        1\times0 + 2\times1 + 1\times2 + 2\times5 & 1\times3 + 2\times(-1) + 1\times1 + 2\times2\\
        4\times0 + 1\times1 - 1\times2 - 4\times5& 4\times3 + 1\times(-1) - 1\times1 - 4\times2\\
    \end{bmatrix}
    \\ {} &=
    \begin{bmatrix}
        14 & 6 \\
        -21 & 2 \\
    \end{bmatrix}
\end{align*}

\textbf{(2a)} \newline
\begin{align*}
A =     
    \begin{bmatrix}
        1 & 1 & -1 & -1 \\
        2 & 5 & -7 & -5 \\
        2 & -1 & 1 & 3 \\
        5 & 2 & -4 & -2 \\
    \end{bmatrix}
\end{align*}

\begin{align*}
    b =     
        \begin{bmatrix}
            1  \\
            -2  \\
            4 \\
            6  \\
        \end{bmatrix}
    \end{align*}

\textbf{(2b)}
Because $A$ is a square matrix and $det(A) \neq 0$, $A$ is nonsingular, 
therefore invertible. And because $A$ is nonsingular, 

\begin{align*}
    A^{-1} = 
    \begin{bmatrix}
        0.5     & -0.167    & 0     & 0.167 \\
        2       & 0.167     & 0.5   & -0.667 \\
        1.75    & -0.25     & 0     & -0.25 \\
        -0.25   & 0.25      & 0.5   & -0.25 \\
    \end{bmatrix}
\end{align*}

\begin{align*}
x = A^{-1}b =
    \left[\begin{array}{cccc}
        0.5     & -0.167    & 0     & 0.167 \\
        2       & 0.167     & 0.5   & -0.667 \\
        1.75    & -0.25     & 0     & -0.25 \\
        -0.25   & 0.25      & 0.5   & -0.25 \\
    \end{array}\right]
    \begin{bmatrix}
        1 \\
        -2 \\
        4 \\
        6 \\
    \end{bmatrix}
    =
    \begin{bmatrix}
        1.83  \\
        -0.33  \\
        0.75 \\
        -0.25  \\
    \end{bmatrix}
\end{align*}

%==============================================================================%
\newpage
\section*{Vector Calculus}
\textbf{(1a)} \newline
Let $g(x) = 1 + e^{-x}$ and $h(g) = g^{-1}$, then we have $f(x) = h(g(x))$.

$f'(x) = h'(g)g'(x) = (-g^{-2})g'(x) = -\frac{1}{(1 + e^{-x})^2} (-e^{-x}) = 
    \frac{1}{1 + e^{-x}} (1 - \frac{1}{1 + e^{-x}}) = \sigma(x)(1 - \sigma(x))$

\textbf{(1b)} \newline
$f'(x) = e^{-\frac{(x - \mu)^2}{2\sigma^2}} (-\frac{2(x - \mu)}{2\sigma^2})
= \frac{\mu - x}{\sigma^2 e^{\frac{(x - \mu)^2}{2\sigma^2}}}$

\textbf{(2a)} \newline

We have 
\begin{align*}
z = x^Tx =
    \left[\begin{array}{cccc}
        x_1 & x_2 & ... & x_D \\
    \end{array}\right]
    \begin{bmatrix}
        x_1     \\
        x_2     \\
        \vdots  \\
        x_D     \\
    \end{bmatrix}
= x_1^2 + x_2^2 + ... + x_D^2
\end{align*}

Applying chain rule results in 

$\nabla_xf = \frac{\partial f}{\partial x} = \frac{\partial f}{\partial z} 
\frac{\partial z}{\partial x} = \frac{1}{1 + z} $

$
\begin{bmatrix}
    \frac{\partial z(x)}{\partial x_1} & \frac{\partial z(x)}{\partial x_x} & ... & \frac{\partial z(x)}{\partial x_D} \\
\end{bmatrix}
= \frac{1}{1 + x_1^2 + x_2^2 + ... + x_D^2} 
\begin{bmatrix}
    2x_1 & 2x_2 & ... & 2x_D \\
\end{bmatrix}
$

Because $z = x^Tx$ is a scalar, $f(z(\textbf{x}))$ is also a scalar. In other 
words, $\textbf{f} : R^D \rightarrow R^1$ maps vector \textbf{x} to a scalar. 
Therefore, the gradient of $f$ with respect to \textbf{x} is a $1 \times D$ 
matrix, which can also be seen as the result of the gradient above.

\textbf{(2b)}
According to chain rule, we have 

$\nabla_xf = \frac{\partial f}{\partial x} = \frac{\partial f}{\partial z} 
\frac{\partial z}{\partial x}$ \tab[1cm] (1)

We also have 
$\frac{\partial f}{\partial z} = -\frac{1}{2e^{z/2}}$ \tab[1cm] (2)

Because $S$ is a symmetric matrix, $S^{-1}$ is also a symmetric matrix, which 
enables us to apply formula (85) in the matrix cookbook as following

$\frac{\partial z}{\partial x} 
= \frac{\partial }{\partial x}(\textbf{x} - \textbf{$\mu$})^TS^{-1}(\textbf{x} - \textbf{$\mu$})
= 2S^{-1}(\textbf{x} - \textbf{$\mu$}) \tab[1cm] (3)
$

$(1),(2),(3) \rightarrow \nabla_xf = -\frac{S^{-1}(\textbf{x} - \textbf{$\mu$})}{e^{z/2}}$

%==============================================================================%
\newpage
\section*{Probability}
\textbf{(1a)} \newline
Let the sample space $\Omega = \{Fair,Unfair\}$, target space 
$\Tau = \{T,F\}$ denoting whether the picked coin is fair or not, and random 
variable $X : \Omega \rightarrow \Tau$.

Assume that the difference between two coins is so small and cannot be 
recognized, then $P(X = T) = P(Fair) = 0.5$.

\textbf{(1b)} \newline
Let us define a sample space $\Omega = \{HH,HT,TH,TT\}$ denoting the outcome of 
two tosses. 
The event we are interested in is whether the first toss is head or not. Let us 
define a random variable $Y$ that maps $\Omega$ to $\Tau = \{H,T\}$, which denotes 
the result of the first toss.

$P(Y = H) = P(Y = H | X = F) + P(Y = H | X = T)$ \newline
$= P_F(H)P(X = T) + P_U(H)P(X = F) = 0.5 \times 0.5 + 0.1 \times 0.5 = 0.3$

\textbf{(1c)} \newline
Applying Bayes's theorem, we have
$P(X = T, HH) = \frac{P(HH | X = T)P(X = T)}{P(HH | X = T)P(X = T) + P(HH | X = F)P(X = F)} 
= \frac{(0.5 \times 0.5) \times 0.5}{(0.5 \times 0.5) \times 0.5 + 0.1 \times 0.1 \times 0.5} \approxeq 0.962$

\textbf{(2a)} \newline
By convention, $p(x) = 1$ and $p(y) = 1$ for discrete random variables with a 
finite number of events.

\textbf{(2b)} \newline
$p(x|Y = y_1) = 0.01 + 0.02 + 0.03 + 0.1 + 0.1 = 0.26$

$p(y|X = x_3) = 0.03 + 0.05 + 0.03 = 0.11$

\textbf{(3a)} \newline
The likelihood function can be written as
$\Pi_{i=1}^nf(x_i;0,\theta) = \Pi_{i=1}^n\frac{1}{\theta - 0} = \frac{1}{\theta^n}$

\textbf{(3a)} \newline
Because the likelihood function is a decreasing function of $\theta$, the 
estimate is smallest when $\theta$ is largest, that is the MLE of $\theta$, $\hat{\theta} = max(x_1, x_2,...,x_n)$.

\textbf{(4a)} \newline
The cost of filtering the mail out is cost of incorrectly predict it as spam; 
$Cost = 10 \times (1 - P) + 0 \times P = 10 \times 0.2 = 2$.

\textbf{(4b)} \newline
The cost of not filtering out the email is cost = $0 \times 0.2 + 1 * 0.8 = 0.8$.

Because it is cheaper not to filter out the email, choose to label the mail as 
non-spam.

\textbf{(4c)} \newline
Applying the threshold will make the cost of filtering out $= 10 \times (1 - \theta)$, 
and the cost of not filtering out $= 1 \times \theta$. The threshold should 
balance of these two costs, that is 

$\theta = 10 \times (1 - \theta) \rightarrow \theta = 10/11$. 

By doing so, we embed the difference in cost of these two decisions in $\theta$. 
The result aligns with our observation from 4a and 4b when false negative cost is 
cheaper than that of false positive. Increasing threshold helps decrease false positive, 
which in turn reduces false positive cost.

\end{document}
